\documentclass[]{article}
\usepackage{lmodern}
\usepackage{amssymb,amsmath}
\usepackage{ifxetex,ifluatex}
\usepackage{fixltx2e} % provides \textsubscript
\ifnum 0\ifxetex 1\fi\ifluatex 1\fi=0 % if pdftex
  \usepackage[T1]{fontenc}
  \usepackage[utf8]{inputenc}
\else % if luatex or xelatex
  \ifxetex
    \usepackage{mathspec}
  \else
    \usepackage{fontspec}
  \fi
  \defaultfontfeatures{Ligatures=TeX,Scale=MatchLowercase}
\fi
% use upquote if available, for straight quotes in verbatim environments
\IfFileExists{upquote.sty}{\usepackage{upquote}}{}
% use microtype if available
\IfFileExists{microtype.sty}{%
\usepackage{microtype}
\UseMicrotypeSet[protrusion]{basicmath} % disable protrusion for tt fonts
}{}
\usepackage[margin=1in]{geometry}
\usepackage{hyperref}
\hypersetup{unicode=true,
            pdftitle={Two column test},
            pdfborder={0 0 0},
            breaklinks=true}
\urlstyle{same}  % don't use monospace font for urls
\usepackage{color}
\usepackage{fancyvrb}
\newcommand{\VerbBar}{|}
\newcommand{\VERB}{\Verb[commandchars=\\\{\}]}
\DefineVerbatimEnvironment{Highlighting}{Verbatim}{commandchars=\\\{\}}
% Add ',fontsize=\small' for more characters per line
\usepackage{framed}
\definecolor{shadecolor}{RGB}{248,248,248}
\newenvironment{Shaded}{\begin{snugshade}}{\end{snugshade}}
\newcommand{\AlertTok}[1]{\textcolor[rgb]{0.94,0.16,0.16}{#1}}
\newcommand{\AnnotationTok}[1]{\textcolor[rgb]{0.56,0.35,0.01}{\textbf{\textit{#1}}}}
\newcommand{\AttributeTok}[1]{\textcolor[rgb]{0.77,0.63,0.00}{#1}}
\newcommand{\BaseNTok}[1]{\textcolor[rgb]{0.00,0.00,0.81}{#1}}
\newcommand{\BuiltInTok}[1]{#1}
\newcommand{\CharTok}[1]{\textcolor[rgb]{0.31,0.60,0.02}{#1}}
\newcommand{\CommentTok}[1]{\textcolor[rgb]{0.56,0.35,0.01}{\textit{#1}}}
\newcommand{\CommentVarTok}[1]{\textcolor[rgb]{0.56,0.35,0.01}{\textbf{\textit{#1}}}}
\newcommand{\ConstantTok}[1]{\textcolor[rgb]{0.00,0.00,0.00}{#1}}
\newcommand{\ControlFlowTok}[1]{\textcolor[rgb]{0.13,0.29,0.53}{\textbf{#1}}}
\newcommand{\DataTypeTok}[1]{\textcolor[rgb]{0.13,0.29,0.53}{#1}}
\newcommand{\DecValTok}[1]{\textcolor[rgb]{0.00,0.00,0.81}{#1}}
\newcommand{\DocumentationTok}[1]{\textcolor[rgb]{0.56,0.35,0.01}{\textbf{\textit{#1}}}}
\newcommand{\ErrorTok}[1]{\textcolor[rgb]{0.64,0.00,0.00}{\textbf{#1}}}
\newcommand{\ExtensionTok}[1]{#1}
\newcommand{\FloatTok}[1]{\textcolor[rgb]{0.00,0.00,0.81}{#1}}
\newcommand{\FunctionTok}[1]{\textcolor[rgb]{0.00,0.00,0.00}{#1}}
\newcommand{\ImportTok}[1]{#1}
\newcommand{\InformationTok}[1]{\textcolor[rgb]{0.56,0.35,0.01}{\textbf{\textit{#1}}}}
\newcommand{\KeywordTok}[1]{\textcolor[rgb]{0.13,0.29,0.53}{\textbf{#1}}}
\newcommand{\NormalTok}[1]{#1}
\newcommand{\OperatorTok}[1]{\textcolor[rgb]{0.81,0.36,0.00}{\textbf{#1}}}
\newcommand{\OtherTok}[1]{\textcolor[rgb]{0.56,0.35,0.01}{#1}}
\newcommand{\PreprocessorTok}[1]{\textcolor[rgb]{0.56,0.35,0.01}{\textit{#1}}}
\newcommand{\RegionMarkerTok}[1]{#1}
\newcommand{\SpecialCharTok}[1]{\textcolor[rgb]{0.00,0.00,0.00}{#1}}
\newcommand{\SpecialStringTok}[1]{\textcolor[rgb]{0.31,0.60,0.02}{#1}}
\newcommand{\StringTok}[1]{\textcolor[rgb]{0.31,0.60,0.02}{#1}}
\newcommand{\VariableTok}[1]{\textcolor[rgb]{0.00,0.00,0.00}{#1}}
\newcommand{\VerbatimStringTok}[1]{\textcolor[rgb]{0.31,0.60,0.02}{#1}}
\newcommand{\WarningTok}[1]{\textcolor[rgb]{0.56,0.35,0.01}{\textbf{\textit{#1}}}}
\usepackage{graphicx,grffile}
\makeatletter
\def\maxwidth{\ifdim\Gin@nat@width>\linewidth\linewidth\else\Gin@nat@width\fi}
\def\maxheight{\ifdim\Gin@nat@height>\textheight\textheight\else\Gin@nat@height\fi}
\makeatother
% Scale images if necessary, so that they will not overflow the page
% margins by default, and it is still possible to overwrite the defaults
% using explicit options in \includegraphics[width, height, ...]{}
\setkeys{Gin}{width=\maxwidth,height=\maxheight,keepaspectratio}
\IfFileExists{parskip.sty}{%
\usepackage{parskip}
}{% else
\setlength{\parindent}{0pt}
\setlength{\parskip}{6pt plus 2pt minus 1pt}
}
\setlength{\emergencystretch}{3em}  % prevent overfull lines
\providecommand{\tightlist}{%
  \setlength{\itemsep}{0pt}\setlength{\parskip}{0pt}}
\setcounter{secnumdepth}{0}
% Redefines (sub)paragraphs to behave more like sections
\ifx\paragraph\undefined\else
\let\oldparagraph\paragraph
\renewcommand{\paragraph}[1]{\oldparagraph{#1}\mbox{}}
\fi
\ifx\subparagraph\undefined\else
\let\oldsubparagraph\subparagraph
\renewcommand{\subparagraph}[1]{\oldsubparagraph{#1}\mbox{}}
\fi

%%% Use protect on footnotes to avoid problems with footnotes in titles
\let\rmarkdownfootnote\footnote%
\def\footnote{\protect\rmarkdownfootnote}

%%% Change title format to be more compact
\usepackage{titling}

% Create subtitle command for use in maketitle
\providecommand{\subtitle}[1]{
  \posttitle{
    \begin{center}\large#1\end{center}
    }
}

\setlength{\droptitle}{-2em}

  \title{Two column test}
    \pretitle{\vspace{\droptitle}\centering\huge}
  \posttitle{\par}
    \author{}
    \preauthor{}\postauthor{}
    \date{}
    \predate{}\postdate{}
  

\begin{document}
\maketitle

This is an example Rmarkdown script that allows for two column
(side-by-side) code environments knitted to both PDF and HTML output.
The key is to include two preamble files --- \texttt{preamble.tex} for
PDF and \texttt{preamble.css} for HTML --- in the YAML. You can then mix
LaTeX and HTML code within the same document, and the knitr engine will
safely ignore any commands that aren't relevant to that output format.
For example, it will ignore any HTML
\texttt{\textless{}div\textgreater{}} tags when knitting to PDF.
Conversely, it will ignore any LaTeX commands
(e.g.~\texttt{\textbackslash{}btwocol},
\texttt{\textbackslash{}etwocol}, and
\texttt{\textbackslash{}columnbreak}) when knitting to HTML.

To illustrate, here comes some code in two columns.

\btwocol

\hypertarget{stata}{%
\paragraph{Stata}\label{stata}}

\begin{Shaded}
\begin{Highlighting}[]
\NormalTok{knitr::include_graphics(}\StringTok{"I:/Thesis_Codes/Thesis_R_Codes/figures/connectivity-plots_25k/Alexandre Laurent (hyper-coupled minority) at 50k rewirings_unserialized.pdf"}\NormalTok{)}
\end{Highlighting}
\end{Shaded}

\columnbreak

\hypertarget{r}{%
\paragraph{R}\label{r}}

\begin{Shaded}
\begin{Highlighting}[]
\NormalTok{knitr}\OperatorTok{::}\KeywordTok{include_graphics}\NormalTok{(}\StringTok{"I:/Thesis_Codes/Thesis_R_Codes/figures/connectivity-plots_25k/Alexandre Laurent (hyper-coupled minority) at 50k rewirings_serialized.pdf"}\NormalTok{)}
\end{Highlighting}
\end{Shaded}

\etwocol

And then we could continue writing. We could still use regular code
chunks spanning the whole page if we wanted.

\begin{Shaded}
\begin{Highlighting}[]
\NormalTok{knitr}\OperatorTok{::}\KeywordTok{include_graphics}\NormalTok{(}\StringTok{"I:/Thesis_Codes/Thesis_R_Codes/figures/connectivity-plots_25k/Alexandre Laurent (hyper-coupled minority) at 50k rewirings_network.pdf"}\NormalTok{)}
\end{Highlighting}
\end{Shaded}

\includegraphics{I:/Thesis_Codes/Thesis_R_Codes/figures/connectivity-plots_25k/Alexandre Laurent (hyper-coupled minority) at 50k rewirings_network.pdf}

And then revert back to the side-by-side format when needed.

\btwocol

\hypertarget{some-people-like-viridis-default}{%
\paragraph{Some people like viridis
default}\label{some-people-like-viridis-default}}

\begin{Shaded}
\begin{Highlighting}[]
\CommentTok{# library(ggplot2)}
\CommentTok{# set.seed(1234)}
\CommentTok{# }
\CommentTok{# p <- ggplot(}
\CommentTok{#   data.frame(x= rnorm(1e4), y = rnorm(1e4)), }
\CommentTok{#   aes(x = x, y = y)}
\CommentTok{#   ) +}
\CommentTok{#   geom_hex() + }
\CommentTok{#   coord_fixed() +}
\CommentTok{#   theme_void()}
\CommentTok{# p + scale_fill_viridis_c()}
\end{Highlighting}
\end{Shaded}

\columnbreak

\hypertarget{others-prefer-viridis-magma}{%
\paragraph{Others prefer viridis
magma}\label{others-prefer-viridis-magma}}

\begin{Shaded}
\begin{Highlighting}[]
\CommentTok{# p + scale_fill_viridis_c(option="A")}
\end{Highlighting}
\end{Shaded}

\etwocol


\end{document}
